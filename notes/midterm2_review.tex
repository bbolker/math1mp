\documentclass[]{tufte-handout}

% ams
\usepackage{amssymb,amsmath}

\usepackage{ifxetex,ifluatex}
\usepackage{fixltx2e} % provides \textsubscript
\ifnum 0\ifxetex 1\fi\ifluatex 1\fi=0 % if pdftex
  \usepackage[T1]{fontenc}
  \usepackage[utf8]{inputenc}
\else % if luatex or xelatex
  \makeatletter
  \@ifpackageloaded{fontspec}{}{\usepackage{fontspec}}
  \makeatother
  \defaultfontfeatures{Ligatures=TeX,Scale=MatchLowercase}
  \makeatletter
  \@ifpackageloaded{soul}{
     \renewcommand\allcapsspacing[1]{{\addfontfeature{LetterSpace=15}#1}}
     \renewcommand\smallcapsspacing[1]{{\addfontfeature{LetterSpace=10}#1}}
   }{}
  \makeatother

\fi

% graphix
\usepackage{graphicx}
\setkeys{Gin}{width=\linewidth,totalheight=\textheight,keepaspectratio}

% booktabs
\usepackage{booktabs}

% url
\usepackage{url}

% hyperref
\usepackage{hyperref}

% units.
\usepackage{units}


\setcounter{secnumdepth}{-1}

% citations

% pandoc syntax highlighting

% longtable

% multiplecol
\usepackage{multicol}

% strikeout
\usepackage[normalem]{ulem}

% morefloats
\usepackage{morefloats}


% tightlist macro required by pandoc >= 1.14
\providecommand{\tightlist}{%
  \setlength{\itemsep}{0pt}\setlength{\parskip}{0pt}}

% title / author / date
\title{1MP3 Midterm 2 Review}
\date{14 November 2019}


\begin{document}

\maketitle




\href{https://docs.python.org/3/tutorial}{Python documentation
reference}.

\begin{center}\rule{0.5\linewidth}{\linethickness}\end{center}

\hypertarget{reading-files}{%
\subsection{reading files}\label{reading-files}}

\begin{itemize}
\tightlist
\item
  \texttt{f\ =\ open(filename,\ "r")}, \texttt{f.close()}
\item
  \texttt{f.read()} reads the \emph{entire file} as a string
\item
  \texttt{f.read(n)} reads the next \texttt{n} characters\\
  (closing and reopening starts from the beginning again)
\item
  reading past the end of a file returns "" (like slicing)
\item
  \texttt{for\ line\ in\ f:} reads a line at a time
\item
  \texttt{f.readline()} reads one line
\end{itemize}

\hypertarget{processing-strings}{%
\subsection{processing strings}\label{processing-strings}}

\begin{itemize}
\tightlist
\item
  strings read from files include \texttt{\textbackslash{}n} (newline)
\item
  \texttt{s.strip()} gets rid of newlines and whitespace
\item
  \texttt{s.split()} splits strings into a list (by spaces, by default)
\item
  \texttt{s.lower()}, \texttt{s.upper()} to convert to lower/uppercase
\item
  \texttt{s.replace(val1,val2)} replaces \texttt{val1} with
  \texttt{val2} in \texttt{s} (e.g.~cleaning punctuation)
\end{itemize}

\hypertarget{sets}{%
\subsection{sets}\label{sets}}

\begin{itemize}
\tightlist
\item
  collections of objects (any type)
\item
  \textbf{unordered} (can't index or slice), \textbf{mutable}
\item
  \textbf{iterable}: can use \texttt{for\ i\ in\ S:}, \texttt{len()},
  \texttt{in}
\item
  define a new set with \texttt{\{"a","b","c"\}} (empty set is
  \texttt{\{\}}) \ldots{}
\item
  \ldots{} or convert from a list/tuple/etc.
  \texttt{set({[}"a","b","c"{]})}
\item
  add new elements with \texttt{S.add("d")}. remove with
  \texttt{remove()}
\item
  \textbf{duplicated elements are silently removed}
\item
  \texttt{.intersection}, \texttt{.union}
\item
  \texttt{.issubset}, comparison operators (\texttt{\textless{}},
  \texttt{\textless{}=} etc.)
\end{itemize}

\hypertarget{dictionaries}{%
\subsection{dictionaries}\label{dictionaries}}

\begin{itemize}
\tightlist
\item
  collections of keys and values
\item
  unordered, mutable, iterable
\item
  keys act like a set
\item
  setup via \texttt{\{"A":1,\ "B":2\}} or
  \texttt{dict({[}{[}"A",1{]},{[}"B",2{]}{]})} or \texttt{dict(A=1,B=2)}
\item
  keys can be any non-mutable type (int, float, tuple)
\item
  values can be anything
\item
  \texttt{for\ i\ in\ d:} iterates over keys; \texttt{in} searches in
  keys
\item
  add \textbf{or replace} a key/value pair: \texttt{d{[}k{]}\ =\ v}
\item
  delete a key/value pair: \texttt{del\ d{[}k{]}}
\item
  \texttt{d{[}k{]}} extracts the value associated with \texttt{k}
\item
  \texttt{d.keys()} returns keys (set-like); \texttt{d.values()} returns
  values (list-like); .\texttt{items()} returns a list-like object
  holding \texttt{(key,value)} tuples
\item
  processing a dictionary (with \texttt{for\ k\ in\ d:} or
  \texttt{for\ k,\ v\ in\ d.items():})
\item
  dictionary inversion
\end{itemize}

\hypertarget{random-number}{%
\subsection{random number}\label{random-number}}

\begin{itemize}
\tightlist
\item
  \texttt{random} and \texttt{numpy.random} modules (similar)
\item
  \texttt{random.seed(102)}: initialize random-number generator (RNG) to
  a known point (for reproducibility)
\item
  \texttt{random.randrange()}: pick one value from a range
\item
  \texttt{random.choice()}: pick one value from a list/tuple
\item
  \texttt{random.random()}: random float uniformly from \([0,1)\)
\item
  \texttt{random.uniform(a,b)}: random float uniformly from \([a,b)\)
\end{itemize}

\hypertarget{numpy-arrays}{%
\subsection{numpy arrays}\label{numpy-arrays}}

\begin{itemize}
\tightlist
\item
  \texttt{np.array()}: from list, tuple, nested lists or tuples
\item
  \texttt{dtype=} argument specifies data type (``float'', ``int32'',
  ``int8'', ``uint8'' etc.)
\item
  \texttt{a.shape} returns a tuple giving dimensions
\item
  \texttt{len(a)} gives length of dimension 0
\item
  also create arrays with \texttt{np.ones()}, \texttt{np.zeros()},
  \texttt{np.arange()}
\item
  \texttt{shape=} argument: tuple specifying dimensions;
  \texttt{np.ones(4)} is the same as \texttt{np.ones((4,))};
  \texttt{np.ones((4,4))} returns a 4 \(\times\) 4 matrix
\item
  \texttt{a.fill(v)} fills array \texttt{a} with value \texttt{v}
\end{itemize}

\hypertarget{slicing-and-indexing-arrays}{%
\subsection{slicing and indexing
arrays}\label{slicing-and-indexing-arrays}}

\begin{itemize}
\tightlist
\item
  indexing: \texttt{a{[}i{]}} or \texttt{a{[}i,j{]}} or
  \texttt{a{[}i,j,k{]}} (depending on dimensions)
\item
  slicing: \texttt{a{[}m:n{]}} or \texttt{a{[}m:n,:{]}} or \ldots;
  \texttt{:} by itself means ``all rows/columns/slices''
\item
  \texttt{a.copy()} to make a copy
\end{itemize}

\hypertarget{reshaping-arrays}{%
\subsection{reshaping arrays}\label{reshaping-arrays}}

\begin{itemize}
\tightlist
\item
  \texttt{a.reshape((r,c))} specifies number of columns (total number of
  elements must match)
\item
  \texttt{a{[}:,np.newaxis{]}} adds a new length-1 dimension
\item
  \texttt{a.flatten()} converts to 1-D
\end{itemize}

\hypertarget{matrices}{%
\subsection{matrices}\label{matrices}}

\begin{itemize}
\tightlist
\item
  \texttt{np.identity}, \texttt{np.eye} for identity matrices
\item
  \textbf{not covered}: linear algebra (\texttt{np.linalg.det},
  \texttt{np.linalg.dot}, \texttt{np.linalg.eig},
  \texttt{np.linalg.inv})
\end{itemize}

\hypertarget{operations}{%
\subsection{operations}\label{operations}}

\begin{itemize}
\tightlist
\item
  all arithmetic (\texttt{+}, \texttt{-}, \texttt{*}, etc.) operates
  \textbf{elementwise} on arrays
\item
  \ldots{} or on array + scalar
\item
  also numpy functions \texttt{np.sin()}, \texttt{np.cos()}, etc.
\item
  \texttt{np.sum()}, \texttt{np.mean()}, \texttt{np.prod()} etc. operate
  on \emph{all elements} by default
\item
  \texttt{axis=i} argument \textbf{collapses dimension i}
  (e.g.~\texttt{np.mean(a,axis=0)} on a 2D array computes mean of each
  column, collapsing rows)
\end{itemize}

\hypertarget{logical-operations}{%
\subsection{logical operations}\label{logical-operations}}

\begin{itemize}
\tightlist
\item
  comparisons (\texttt{\textgreater{}}, \texttt{==} etc.) work
  elementwise, producing a \texttt{bool} array
\item
  \texttt{np.logical\_and()}, \texttt{np.logical\_or()},
  \texttt{np.logical\_not()}
\item
  \texttt{a{[}b{]}} selects the elements of \texttt{a} for which bool
  array \texttt{b} is \texttt{True}
\item
  e.g.~\texttt{a{[}a\textgreater{}0{]}} selects positive elements
\end{itemize}

\hypertarget{numerics}{%
\subsection{numerics}\label{numerics}}

\begin{itemize}
\tightlist
\item
  \texttt{numpy} integers: for an \(n\)-bit integer, one is the sign
  bit, so the maximum positive value is \(2^{n-1}\); maximum negative is
  \(-2^n\)
\item
  going out of bounds ``wraps around''
\item
  plain (not \texttt{numpy}) integers are special, won't overflow
\item
  floating-point: \textbf{often experience rounding error}. Don't assume
  math works exactly.
\item
  use \texttt{np.isclose()} or \texttt{math.isclose()} to test
  near-equality
\item
  overflow

  \begin{itemize}
  \tightlist
  \item
    for regular (64-bit) floats, values greater than
    \(\approx 2^{2^{10}} \approx 10^{308}\) become \texttt{inf}
  \item
    values less than \(\approx -10^{308}\) become \texttt{-inf}
  \item
    undefined operations (e.g.~\texttt{inf-inf}, \texttt{inf/inf})
    become \texttt{nan} (not a number)
  \end{itemize}
\item
  underflow

  \begin{itemize}
  \tightlist
  \item
    values less than \(\approx 2^{-2^{10}} \approx 10^{-308}\) become 0
  \item
    adding \emph{relatively} much smaller numbers (i.e.~\(a+b\) where
    \(b/a \lt 2^{-53} \approx 10^{-16}\)), they disappear:
    e.g.~\(1+x==1\) if \(x\) is very small
  \end{itemize}
\end{itemize}

This appears on the test:

\begin{quote}
Some helpful numbers: \(2^7=128\); \(2^8=256\);
\(2^{2^{10}} \approx 10^{308}\); \(2^{-53} \approx 10^{-16}\).
\end{quote}



\end{document}

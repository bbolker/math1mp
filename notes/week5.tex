\documentclass[]{tufte-handout}

% ams
\usepackage{amssymb,amsmath}

\usepackage{ifxetex,ifluatex}
\usepackage{fixltx2e} % provides \textsubscript
\ifnum 0\ifxetex 1\fi\ifluatex 1\fi=0 % if pdftex
  \usepackage[T1]{fontenc}
  \usepackage[utf8]{inputenc}
\else % if luatex or xelatex
  \makeatletter
  \@ifpackageloaded{fontspec}{}{\usepackage{fontspec}}
  \makeatother
  \defaultfontfeatures{Ligatures=TeX,Scale=MatchLowercase}
  \makeatletter
  \@ifpackageloaded{soul}{
     \renewcommand\allcapsspacing[1]{{\addfontfeature{LetterSpace=15}#1}}
     \renewcommand\smallcapsspacing[1]{{\addfontfeature{LetterSpace=10}#1}}
   }{}
  \makeatother

\fi

% graphix
\usepackage{graphicx}
\setkeys{Gin}{width=\linewidth,totalheight=\textheight,keepaspectratio}

% booktabs
\usepackage{booktabs}

% url
\usepackage{url}

% hyperref
\usepackage{hyperref}

% units.
\usepackage{units}


\setcounter{secnumdepth}{-1}

% citations

% pandoc syntax highlighting
\usepackage{color}
\usepackage{fancyvrb}
\newcommand{\VerbBar}{|}
\newcommand{\VERB}{\Verb[commandchars=\\\{\}]}
\DefineVerbatimEnvironment{Highlighting}{Verbatim}{commandchars=\\\{\}}
% Add ',fontsize=\small' for more characters per line
\newenvironment{Shaded}{}{}
\newcommand{\KeywordTok}[1]{\textcolor[rgb]{0.00,0.44,0.13}{\textbf{{#1}}}}
\newcommand{\DataTypeTok}[1]{\textcolor[rgb]{0.56,0.13,0.00}{{#1}}}
\newcommand{\DecValTok}[1]{\textcolor[rgb]{0.25,0.63,0.44}{{#1}}}
\newcommand{\BaseNTok}[1]{\textcolor[rgb]{0.25,0.63,0.44}{{#1}}}
\newcommand{\FloatTok}[1]{\textcolor[rgb]{0.25,0.63,0.44}{{#1}}}
\newcommand{\ConstantTok}[1]{\textcolor[rgb]{0.53,0.00,0.00}{{#1}}}
\newcommand{\CharTok}[1]{\textcolor[rgb]{0.25,0.44,0.63}{{#1}}}
\newcommand{\SpecialCharTok}[1]{\textcolor[rgb]{0.25,0.44,0.63}{{#1}}}
\newcommand{\StringTok}[1]{\textcolor[rgb]{0.25,0.44,0.63}{{#1}}}
\newcommand{\VerbatimStringTok}[1]{\textcolor[rgb]{0.25,0.44,0.63}{{#1}}}
\newcommand{\SpecialStringTok}[1]{\textcolor[rgb]{0.73,0.40,0.53}{{#1}}}
\newcommand{\ImportTok}[1]{{#1}}
\newcommand{\CommentTok}[1]{\textcolor[rgb]{0.38,0.63,0.69}{\textit{{#1}}}}
\newcommand{\DocumentationTok}[1]{\textcolor[rgb]{0.73,0.13,0.13}{\textit{{#1}}}}
\newcommand{\AnnotationTok}[1]{\textcolor[rgb]{0.38,0.63,0.69}{\textbf{\textit{{#1}}}}}
\newcommand{\CommentVarTok}[1]{\textcolor[rgb]{0.38,0.63,0.69}{\textbf{\textit{{#1}}}}}
\newcommand{\OtherTok}[1]{\textcolor[rgb]{0.00,0.44,0.13}{{#1}}}
\newcommand{\FunctionTok}[1]{\textcolor[rgb]{0.02,0.16,0.49}{{#1}}}
\newcommand{\VariableTok}[1]{\textcolor[rgb]{0.10,0.09,0.49}{{#1}}}
\newcommand{\ControlFlowTok}[1]{\textcolor[rgb]{0.00,0.44,0.13}{\textbf{{#1}}}}
\newcommand{\OperatorTok}[1]{\textcolor[rgb]{0.40,0.40,0.40}{{#1}}}
\newcommand{\BuiltInTok}[1]{{#1}}
\newcommand{\ExtensionTok}[1]{{#1}}
\newcommand{\PreprocessorTok}[1]{\textcolor[rgb]{0.74,0.48,0.00}{{#1}}}
\newcommand{\AttributeTok}[1]{\textcolor[rgb]{0.49,0.56,0.16}{{#1}}}
\newcommand{\RegionMarkerTok}[1]{{#1}}
\newcommand{\InformationTok}[1]{\textcolor[rgb]{0.38,0.63,0.69}{\textbf{\textit{{#1}}}}}
\newcommand{\WarningTok}[1]{\textcolor[rgb]{0.38,0.63,0.69}{\textbf{\textit{{#1}}}}}
\newcommand{\AlertTok}[1]{\textcolor[rgb]{1.00,0.00,0.00}{\textbf{{#1}}}}
\newcommand{\ErrorTok}[1]{\textcolor[rgb]{1.00,0.00,0.00}{\textbf{{#1}}}}
\newcommand{\NormalTok}[1]{{#1}}

% longtable

% multiplecol
\usepackage{multicol}

% strikeout
\usepackage[normalem]{ulem}

% morefloats
\usepackage{morefloats}


% tightlist macro required by pandoc >= 1.14
\providecommand{\tightlist}{%
  \setlength{\itemsep}{0pt}\setlength{\parskip}{0pt}}

% title / author / date
\title{examples from calc 1}
\author{Ben Bolker}
\date{03 October 2019}


\begin{document}

\maketitle




\subsection{Tuples}\label{tuples}

\begin{itemize}
\tightlist
\item
  simple; \textbf{non-mutable} version of lists
\item
  faster, safer
\item
  can be expressed as \texttt{x,\ y,\ z} (or \texttt{(x,y,z)}, probably
  clearer)
\item
  empty tuple: \texttt{()}
\item
  tuple with one element: \texttt{(x,)}
\item
  can do many of the same things as with lists
\end{itemize}

\begin{Shaded}
\begin{Highlighting}[]
\NormalTok{x }\OperatorTok{=} \NormalTok{(}\DecValTok{1}\NormalTok{,}\DecValTok{4}\NormalTok{,}\StringTok{"a"}\NormalTok{,}\DecValTok{3}\NormalTok{)}
\BuiltInTok{print}\NormalTok{(x[}\DecValTok{1}\NormalTok{])   }\CommentTok{## indexing}
\end{Highlighting}
\end{Shaded}

\begin{verbatim}
## 4
\end{verbatim}

\begin{Shaded}
\begin{Highlighting}[]
\BuiltInTok{print}\NormalTok{(x[}\DecValTok{2}\NormalTok{:])  }\CommentTok{## slicing}
\end{Highlighting}
\end{Shaded}

\begin{verbatim}
## ('a', 3)
\end{verbatim}

\begin{Shaded}
\begin{Highlighting}[]
\BuiltInTok{print}\NormalTok{(x}\OperatorTok{+}\NormalTok{(}\DecValTok{3}\NormalTok{,)) }\CommentTok{## appending}
\end{Highlighting}
\end{Shaded}

\begin{verbatim}
## (1, 4, 'a', 3, 3)
\end{verbatim}

\begin{Shaded}
\begin{Highlighting}[]
\BuiltInTok{print}\NormalTok{(x[:}\DecValTok{2}\NormalTok{] }\OperatorTok{+} \NormalTok{(}\DecValTok{3}\NormalTok{,) }\OperatorTok{+} \NormalTok{x[}\DecValTok{2}\NormalTok{:]) }\CommentTok{## insertion in the middle}
\end{Highlighting}
\end{Shaded}

\begin{verbatim}
## (1, 4, 3, 'a', 3)
\end{verbatim}

\begin{Shaded}
\begin{Highlighting}[]
\NormalTok{x.index(}\DecValTok{4}\NormalTok{)    }\CommentTok{## indexing}
\end{Highlighting}
\end{Shaded}

\begin{verbatim}
## 1
\end{verbatim}

\begin{Shaded}
\begin{Highlighting}[]
\CommentTok{"z"} \KeywordTok{in} \NormalTok{x      }\CommentTok{## looking for stuff}
\end{Highlighting}
\end{Shaded}

\begin{verbatim}
## False
\end{verbatim}

\begin{Shaded}
\begin{Highlighting}[]
\NormalTok{x.count(}\DecValTok{4}\NormalTok{)    }\CommentTok{## count}
\end{Highlighting}
\end{Shaded}

\begin{verbatim}
## 1
\end{verbatim}

\begin{itemize}
\tightlist
\item
  you \emph{can't} modify the existing tuple at all (deletion,
  modification)
\item
  unpacking: \texttt{x,y,z\ =\ t}
\item
  swapping: \texttt{(a,b)\ =\ (b,a)}
\item
  useful as the return value of functions; safe, and can be unpacked
\item
  convert to/from lists (\texttt{tuple()}, \texttt{list()})
\end{itemize}

\begin{Shaded}
\begin{Highlighting}[]
\NormalTok{x }\OperatorTok{=} \NormalTok{(}\DecValTok{1}\NormalTok{,}\DecValTok{2}\NormalTok{,}\DecValTok{3}\NormalTok{)}
\KeywordTok{def} \NormalTok{modify(x):}
    \NormalTok{y }\OperatorTok{=} \BuiltInTok{list}\NormalTok{(x)}
    \NormalTok{y[}\DecValTok{0}\NormalTok{] }\OperatorTok{=} \StringTok{"a"}
    \ControlFlowTok{return}\NormalTok{(}\BuiltInTok{tuple}\NormalTok{(y))}

\BuiltInTok{print}\NormalTok{(modify(x))}
\end{Highlighting}
\end{Shaded}

\begin{verbatim}
## ('a', 2, 3)
\end{verbatim}

\begin{Shaded}
\begin{Highlighting}[]
\BuiltInTok{print}\NormalTok{(x)}
\end{Highlighting}
\end{Shaded}

\begin{verbatim}
## (1, 2, 3)
\end{verbatim}

\subsection{reminders/clarifications}\label{remindersclarifications}

\begin{itemize}
\tightlist
\item
  parentheses (\texttt{()}) vs square brackets (\texttt{{[}{]}})
\item
  \textbf{square brackets}

  \begin{itemize}
  \tightlist
  \item
    indexing (lists or strings or tuples): \texttt{x{[}5{]}}
  \item
    slicing (lists or strings or tuples): \texttt{x{[}5:7{]}}
  \item
    defining lists: \texttt{{[}1,2,3{]}}
  \end{itemize}
\item
  \textbf{parentheses}

  \begin{itemize}
  \tightlist
  \item
    order of operations: \texttt{(1+2)*3},
    \texttt{a\ and\ (not\ b\ or\ c)}
  \item
    calling functions: \texttt{len(x)}, \texttt{range(5)},
    \texttt{print("hello")}
  \item
    calling methods: \texttt{x.sort()}, \texttt{x.append(4)}
  \item
    defining functions: \texttt{def\ f(x1,x2,x3):}
  \item
    returning values from functions: \texttt{return(x)} (*)
  \item
    defining tuples: \texttt{()}, \texttt{(1,)}, \texttt{(2,3)} (*)
  \end{itemize}
\end{itemize}

``\emph{" actually (mostly) }optional*: see
\href{https://stackoverflow.com/questions/4978567/should-a-return-statement-have-parentheses}{here}

\subsection{Root-finding methods}\label{root-finding-methods}

\begin{itemize}
\tightlist
\item
  Assume that \(f(x)\) is a continuous function on the real numbers.
\item
  Suppose that \(a < b\)
\item
  Suppose \emph{endpoints are of opposite signs}:\\
  \(f(a)<0\) and \(f(b)> 0\) \textbf{or} \(f(b)<0\) and \(f(a)>0\)
\item
  (or \(f(a)\cdot f(b) <0\))
\item
  By the Intermediate Value Theorem, there is some number \(c\) between
  \(a\) and \(b\) with \(f(c) = 0\); this is called a \textbf{root} of
  the function \(f\)
\end{itemize}

We will use three methods (Grid, Bisection, and Newton's method) to
approximate such a number \(c\). (There may be more than one root of
\(f\) in the interval between \(a\) and \(b\).)

\subsection{Example}\label{example}

\begin{itemize}
\tightlist
\item
  We'll use \(\exp(x)-x-3/2\) as an example
\item
  impossible to do analytically!
\item
  value at 0 = -3/2
\item
  value at 1 = \(\exp(1)-1-3/2 \approx 2.78 - 2.5 = 0.28\)
\end{itemize}

\subsection{Grid Method}\label{grid-method}

\begin{itemize}
\tightlist
\item
  Break the interval \([a, b]\) into \(n\) subintervals of equal sizes,
  having endpoints \[
  x_0 = a, x_1 , \dots , x_{n−1} , x_n = b \quad .
  \]
\item
  Compute \(f(x_0), f(x_1), \dots , f(x_n)\)
\item
  Find the index \(i\) such that \(f(x_i)\) is closest to 0 and use this
  to approximate a root of \(f\) in the interval \([a, b]\).
\item
  \textbf{Project:} Create a function
  \texttt{grid\_search(f,\ a,\ b,\ n)} that implements the grid method.
\end{itemize}

\subsection{Bisection Method}\label{bisection-method}

\begin{itemize}
\tightlist
\item
  Bisect the interval \([a, b]\) into two equal subintervals \([a, m]\),
  \([m, b]\), where \(m = (a + b)/2\).
\item
  If \(f(a)\) and \(f(m)\) have opposite signs, then there will be a
  root in \([a, m]\). Otherwise, there will be a root in \([m, b]\).
\item
  Bisect this subinterval (\([a, m]\) in the former case, \([m, b]\) in
  the latter), and continue bisecting until the subinterval is small.
\item
  A root of \(f\) will be located in this small subinterval.
\item
  \textbf{Project:} Create a function \texttt{bisect(f,\ a,\ b,\ tol)}
  that approximates a root of \texttt{f} in the interval
  \texttt{{[}a,\ b{]}} with an error of at most \texttt{tol}.
\end{itemize}

\subsection{Newton's Method}\label{newtons-method}

\begin{itemize}
\tightlist
\item
  Suppose we know the derivative (\emph{gradient}) \(df/dx=f'(x)\) as
  well as \(f(x)\)
\item
  For a given starting value \(x_0\), guess the position of the root
  according to \(x_1 = f(x_0)-\frac{x_0}{f'(x_0)}\).
\item
  Repeat until we are within tolerance of the root (\(|f(x)|\) is small
\item
  \textbf{Project:} Create a function
  \texttt{newton(f,\ grad,\ x0,\ tol,\ nmax)} that approximates a root
  of \texttt{f} with an error of at most \texttt{tol}, taking no more
  than \texttt{nmax} steps.
\end{itemize}



\end{document}

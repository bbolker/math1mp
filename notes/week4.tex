\documentclass[]{tufte-handout}

% ams
\usepackage{amssymb,amsmath}

\usepackage{ifxetex,ifluatex}
\usepackage{fixltx2e} % provides \textsubscript
\ifnum 0\ifxetex 1\fi\ifluatex 1\fi=0 % if pdftex
  \usepackage[T1]{fontenc}
  \usepackage[utf8]{inputenc}
\else % if luatex or xelatex
  \makeatletter
  \@ifpackageloaded{fontspec}{}{\usepackage{fontspec}}
  \makeatother
  \defaultfontfeatures{Ligatures=TeX,Scale=MatchLowercase}
  \makeatletter
  \@ifpackageloaded{soul}{
     \renewcommand\allcapsspacing[1]{{\addfontfeature{LetterSpace=15}#1}}
     \renewcommand\smallcapsspacing[1]{{\addfontfeature{LetterSpace=10}#1}}
   }{}
  \makeatother

\fi

% graphix
\usepackage{graphicx}
\setkeys{Gin}{width=\linewidth,totalheight=\textheight,keepaspectratio}

% booktabs
\usepackage{booktabs}

% url
\usepackage{url}

% hyperref
\usepackage{hyperref}

% units.
\usepackage{units}


\setcounter{secnumdepth}{-1}

% citations

% pandoc syntax highlighting
\usepackage{color}
\usepackage{fancyvrb}
\newcommand{\VerbBar}{|}
\newcommand{\VERB}{\Verb[commandchars=\\\{\}]}
\DefineVerbatimEnvironment{Highlighting}{Verbatim}{commandchars=\\\{\}}
% Add ',fontsize=\small' for more characters per line
\newenvironment{Shaded}{}{}
\newcommand{\KeywordTok}[1]{\textcolor[rgb]{0.00,0.44,0.13}{\textbf{{#1}}}}
\newcommand{\DataTypeTok}[1]{\textcolor[rgb]{0.56,0.13,0.00}{{#1}}}
\newcommand{\DecValTok}[1]{\textcolor[rgb]{0.25,0.63,0.44}{{#1}}}
\newcommand{\BaseNTok}[1]{\textcolor[rgb]{0.25,0.63,0.44}{{#1}}}
\newcommand{\FloatTok}[1]{\textcolor[rgb]{0.25,0.63,0.44}{{#1}}}
\newcommand{\ConstantTok}[1]{\textcolor[rgb]{0.53,0.00,0.00}{{#1}}}
\newcommand{\CharTok}[1]{\textcolor[rgb]{0.25,0.44,0.63}{{#1}}}
\newcommand{\SpecialCharTok}[1]{\textcolor[rgb]{0.25,0.44,0.63}{{#1}}}
\newcommand{\StringTok}[1]{\textcolor[rgb]{0.25,0.44,0.63}{{#1}}}
\newcommand{\VerbatimStringTok}[1]{\textcolor[rgb]{0.25,0.44,0.63}{{#1}}}
\newcommand{\SpecialStringTok}[1]{\textcolor[rgb]{0.73,0.40,0.53}{{#1}}}
\newcommand{\ImportTok}[1]{{#1}}
\newcommand{\CommentTok}[1]{\textcolor[rgb]{0.38,0.63,0.69}{\textit{{#1}}}}
\newcommand{\DocumentationTok}[1]{\textcolor[rgb]{0.73,0.13,0.13}{\textit{{#1}}}}
\newcommand{\AnnotationTok}[1]{\textcolor[rgb]{0.38,0.63,0.69}{\textbf{\textit{{#1}}}}}
\newcommand{\CommentVarTok}[1]{\textcolor[rgb]{0.38,0.63,0.69}{\textbf{\textit{{#1}}}}}
\newcommand{\OtherTok}[1]{\textcolor[rgb]{0.00,0.44,0.13}{{#1}}}
\newcommand{\FunctionTok}[1]{\textcolor[rgb]{0.02,0.16,0.49}{{#1}}}
\newcommand{\VariableTok}[1]{\textcolor[rgb]{0.10,0.09,0.49}{{#1}}}
\newcommand{\ControlFlowTok}[1]{\textcolor[rgb]{0.00,0.44,0.13}{\textbf{{#1}}}}
\newcommand{\OperatorTok}[1]{\textcolor[rgb]{0.40,0.40,0.40}{{#1}}}
\newcommand{\BuiltInTok}[1]{{#1}}
\newcommand{\ExtensionTok}[1]{{#1}}
\newcommand{\PreprocessorTok}[1]{\textcolor[rgb]{0.74,0.48,0.00}{{#1}}}
\newcommand{\AttributeTok}[1]{\textcolor[rgb]{0.49,0.56,0.16}{{#1}}}
\newcommand{\RegionMarkerTok}[1]{{#1}}
\newcommand{\InformationTok}[1]{\textcolor[rgb]{0.38,0.63,0.69}{\textbf{\textit{{#1}}}}}
\newcommand{\WarningTok}[1]{\textcolor[rgb]{0.38,0.63,0.69}{\textbf{\textit{{#1}}}}}
\newcommand{\AlertTok}[1]{\textcolor[rgb]{1.00,0.00,0.00}{\textbf{{#1}}}}
\newcommand{\ErrorTok}[1]{\textcolor[rgb]{1.00,0.00,0.00}{\textbf{{#1}}}}
\newcommand{\NormalTok}[1]{{#1}}

% longtable

% multiplecol
\usepackage{multicol}

% strikeout
\usepackage[normalem]{ulem}

% morefloats
\usepackage{morefloats}


% tightlist macro required by pandoc >= 1.14
\providecommand{\tightlist}{%
  \setlength{\itemsep}{0pt}\setlength{\parskip}{0pt}}

% title / author / date
\title{modules, more functions, hexadecimal, tuples}
\author{Ben Bolker}
\date{23 September 2019}


\begin{document}

\maketitle




\section{Modules}\label{modules}

Collections of functions you might want to use.

\subsection{importing}\label{importing}

\begin{itemize}
\tightlist
\item
  use \texttt{import} to make functions inside modules available
\item
  refer to functions via module prefix
\item
  \texttt{import\ VeryLongModuleName\ as\ vlmn}: use abbreviation
\item
  can import just one or two functions:
  \texttt{from\ math\ import\ sqrt,\ log}
\item
  can import everything (but usually don't):
  \texttt{from\ \textless{}module\textgreater{}\ import\ *}
\item
  can import \emph{your own modules} (i.e., functions in a \texttt{.py}
  file)
\end{itemize}

\subsection{finding out about modules}\label{finding-out-about-modules}

\begin{itemize}
\tightlist
\item
  \texttt{help("modulename")}
\item
  \href{https://docs.python.org/3/py-modindex.html}{official modules}
\item
  \href{https://wiki.python.org/moin/UsefulModules}{list of useful
  modules}
\item
  some modules we will definitely be using:

  \begin{itemize}
  \tightlist
  \item
    \texttt{math}: basic math functions
  \item
    \texttt{matplotlib}: drawing pictures
  \item
    \texttt{random}: picking random numbers
  \item
    \texttt{numpy}: numerical computation\\
    (including linear algebra and some calculus)
  \item
    \texttt{pandas}: data analysis
  \end{itemize}
\item
  more tangential but maybe used:

  \begin{itemize}
  \tightlist
  \item
    \texttt{nose}: code testing framework
  \item
    \texttt{scipy}: even more scientific computing tools
  \item
    \texttt{cmath}: math functions handling complex numbers
  \item
    \texttt{re}: regular expressions
  \item
    \texttt{sympy}: symbolic computation
  \item
    \texttt{timeit}: how long does my code take?
  \end{itemize}
\end{itemize}

\subsection{Functions calling
functions}\label{functions-calling-functions}

\begin{itemize}
\tightlist
\item
  You can pass anything to a function as an argument (even a function!)
\end{itemize}

\begin{Shaded}
\begin{Highlighting}[]
\KeywordTok{def} \NormalTok{repeat_fun(f,startval,n):}
    \CommentTok{"""Given a function f and a starting value startval,}
\CommentTok{    apply the function n times (each time using the previous}
\CommentTok{    result as input)}
\CommentTok{    """}
    \NormalTok{y }\OperatorTok{=} \NormalTok{startval}
    \ControlFlowTok{for} \NormalTok{i }\KeywordTok{in} \BuiltInTok{range}\NormalTok{(n):}
        \NormalTok{y}\OperatorTok{=}\NormalTok{f(y)}
    \ControlFlowTok{return}\NormalTok{(y)}

\KeywordTok{def} \NormalTok{sqr(x):}
    \ControlFlowTok{return}\NormalTok{(x}\OperatorTok{*}\NormalTok{x)}

\NormalTok{repeat_fun(sqr,}\DecValTok{3}\NormalTok{,}\DecValTok{3}\NormalTok{)}
\end{Highlighting}
\end{Shaded}

\begin{verbatim}
## 6561
\end{verbatim}

\subsection{Function composition}\label{function-composition}

\begin{itemize}
\tightlist
\item
  Mathematically this kind of example is called \textbf{composition} of
  a function with itself (see
  \href{https://en.wikipedia.org/wiki/Function_composition}{Wikipedia}
\item
  in math notation: \((g\circ f)(x) = f(g(x))\)
\item
  (notation for multiple composition of a function with itself
  \href{https://math.stackexchange.com/questions/926247/notation-for-repeated-composition-of-functions}{is
  harder})
\item
  write a function \texttt{compose\_funs(f,g)}
\end{itemize}

\subsection{Recursion}\label{recursion}

Functions can even call themselves! This is like mathematical
\href{https://en.wikipedia.org/wiki/Mathematical_induction}{induction}.

\begin{Shaded}
\begin{Highlighting}[]
\KeywordTok{def} \NormalTok{factorial(x):}
    \ControlFlowTok{if} \NormalTok{(x}\OperatorTok{==}\DecValTok{1}\NormalTok{):}
        \ControlFlowTok{return}\NormalTok{(}\DecValTok{1}\NormalTok{)}
    \ControlFlowTok{return}\NormalTok{(x}\OperatorTok{*}\NormalTok{factorial(x}\DecValTok{-1}\NormalTok{))}

\NormalTok{factorial(}\DecValTok{5}\NormalTok{)}
\end{Highlighting}
\end{Shaded}

\begin{verbatim}
## 120
\end{verbatim}

\subsection{Scope}\label{scope}

\begin{itemize}
\tightlist
\item
  Where does Python look for things?
\item
  What happens here?
\end{itemize}

\begin{Shaded}
\begin{Highlighting}[]
\NormalTok{z }\OperatorTok{=} \DecValTok{1}
\KeywordTok{def} \NormalTok{add_z(x):}
    \ControlFlowTok{return}\NormalTok{(x}\OperatorTok{+}\NormalTok{z)}

\NormalTok{add_z(z)}
\end{Highlighting}
\end{Shaded}

\begin{verbatim}
## 2
\end{verbatim}

\subsection{Scoping rules}\label{scoping-rules}

\textbf{LEGB} (Local, Enclosing, Global, Built-in) - \emph{Local}:
symbols defined in the function, and arguments - \emph{Enclosing}:
symbols defined \emph{in the function within which this function was
defined} - \emph{Global}: elsewhere in the file/module -
\emph{Built-in}: Python keywords

\subsection{Hexadecimal/Decimal
conversion}\label{hexadecimaldecimal-conversion}

\begin{itemize}
\tightlist
\item
  The \textbf{hexadecimal} (or ``base 16'') numeral system uses sixteen
  distinct digits to represent integers.
\item
  The digits used are: 0, 1, 2, 3, 4, 5, 6, 7, 8, 9, a, b, c, d, e, f .
\item
  The decimal value of the digit \texttt{a} is 10, \texttt{b} is 11,
  etc.
\item
  The hexadecimal number 2c is equal to \(12 ∗ 16^0 + 2 ∗ 16^1 = 44\)
  (base 10).
\item
  Similarly, \texttt{2be13} is equal to 179731 since \[
  179731 = 3 ∗ 16^0 + 1 ∗ 16^1 + 14 ∗ 16^2 + 11 ∗ 16^3 + 2 ∗ 16^4 \quad.
  \]
\item
  The number 1020304 in hexadecimal is f9190. This can be verified by
  expanding f9190 as \[
  0 ∗ 16^0 + 9 ∗ 16^1 + 1 ∗ 16^2 + 9 ∗ 16^3 + 15 ∗ 16^4 ,
  \] which is equal to \(1020304_{10}\)
\end{itemize}

\subsection{Problem}\label{problem}

\begin{itemize}
\tightlist
\item
  Write Python code that takes as input from the console two strings
  that represent numbers in the hexadecimal system.
\item
  The program should should print out the representations of these
  numbers in base 10, and also print a string that represents the sum of
  these numbers in hexadecimal.
\end{itemize}

\subsection{High level description of the
algorithm}\label{high-level-description-of-the-algorithm}

\begin{enumerate}
\def\labelenumi{\arabic{enumi}.}
\tightlist
\item
  Input the two strings from the console.
\item
  Convert each string into a base 10 number.
\item
  Print out these two numbers.
\item
  Convert the sum of these two numbers into hexadecimal.
\item
  Print out this hexadecimal number.
\end{enumerate}

\begin{center}\rule{0.5\linewidth}{\linethickness}\end{center}

\begin{itemize}
\tightlist
\item
  For Step 1, use the \texttt{input()} function.
\item
  Create a function \texttt{get\_hex\_string()} that gets a string from
  the console that represents a hexadecimal number and returns that
  string.
\item
  Should it check to see if it is a legal string, i.e., only uses 0 − 9,
  and a − f ?
\end{itemize}

\subsection{convert hexadecimal into
decimal}\label{convert-hexadecimal-into-decimal}

\begin{itemize}
\tightlist
\item
  if an integer is represented in hexadecimal by the string of length
  \(n\) \texttt{word} \(= h_{n−1} h_{n−2} \dots h_1 h_0\)
\item
  then it is equal to the number: \[
  h_{n−1} * 16^{n−1} + h_{n−2} * 16^{n−2} + \dots + h_0 * 16^0 \quad .
  \]
\item
  So to convert word into decimal, we can iterate over each digit in
  \texttt{word} to produce the required value.
\item
  Note that the \(j^{\textrm{th}}\) term in the above sum is equal to
  \$h\_\{n−j−1\} ∗ 16\^{}\{n−j−1\} , with \(j = 0, \dots, n − 1\) and
  that the digit \(h_{n−j}\) is just \texttt{word{[}j{]}}.
\item
  \textbf{next step}: Create a function
  \texttt{hex\_to\_decimal(hex\_String)} with string argument
  \texttt{hex\_string} that will returns the value of the base-10
  integer this string represents in hexadecimal \ldots{}
\end{itemize}

\subsection{convert to hexadecimal}\label{convert-to-hexadecimal}

\begin{itemize}
\tightlist
\item
  To find the hexadecimal digits \(h_k h_{k−1} \dots h_1 h_0\) of the
  non-negative base-10 integer \texttt{num} we use \texttt{//} and
  \texttt{\%}.

  \begin{itemize}
  \tightlist
  \item
    \texttt{h{[}0{]}\ =\ num\ \%\ 16}
  \item
    \texttt{h{[}1{]}\ =\ (num\ //\ 16)\ \%\ 16}
  \item
    \texttt{h{[}2{]}\ =\ (num\ //\ 16**2\ )\ \%\ 16}
  \item
    \ldots{}
  \item
    \texttt{h{[}i{]}\ =\ (num//\ 16**i\ )\ \%\ 16}
  \end{itemize}
\item
  (But we can do this more easily as a variation of the
  \textbf{coin-counting problem} \ldots{}
\item
  Q: How do we decide when to stop?
\item
  \textbf{next step}: Produce a function \texttt{decimal\_to\_hex(num)}
  that computes the hexadecimal representation of the int \texttt{num}
  and returns this as a string.
\item
  To finish, use these functions to produce the final result.
\end{itemize}



\end{document}

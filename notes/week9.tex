\documentclass[]{tufte-handout}

% ams
\usepackage{amssymb,amsmath}

\usepackage{ifxetex,ifluatex}
\usepackage{fixltx2e} % provides \textsubscript
\ifnum 0\ifxetex 1\fi\ifluatex 1\fi=0 % if pdftex
  \usepackage[T1]{fontenc}
  \usepackage[utf8]{inputenc}
\else % if luatex or xelatex
  \makeatletter
  \@ifpackageloaded{fontspec}{}{\usepackage{fontspec}}
  \makeatother
  \defaultfontfeatures{Ligatures=TeX,Scale=MatchLowercase}
  \makeatletter
  \@ifpackageloaded{soul}{
     \renewcommand\allcapsspacing[1]{{\addfontfeature{LetterSpace=15}#1}}
     \renewcommand\smallcapsspacing[1]{{\addfontfeature{LetterSpace=10}#1}}
   }{}
  \makeatother

\fi

% graphix
\usepackage{graphicx}
\setkeys{Gin}{width=\linewidth,totalheight=\textheight,keepaspectratio}

% booktabs
\usepackage{booktabs}

% url
\usepackage{url}

% hyperref
\usepackage{hyperref}

% units.
\usepackage{units}


\setcounter{secnumdepth}{-1}

% citations

% pandoc syntax highlighting
\usepackage{color}
\usepackage{fancyvrb}
\newcommand{\VerbBar}{|}
\newcommand{\VERB}{\Verb[commandchars=\\\{\}]}
\DefineVerbatimEnvironment{Highlighting}{Verbatim}{commandchars=\\\{\}}
% Add ',fontsize=\small' for more characters per line
\newenvironment{Shaded}{}{}
\newcommand{\AlertTok}[1]{\textcolor[rgb]{1.00,0.00,0.00}{\textbf{#1}}}
\newcommand{\AnnotationTok}[1]{\textcolor[rgb]{0.38,0.63,0.69}{\textbf{\textit{#1}}}}
\newcommand{\AttributeTok}[1]{\textcolor[rgb]{0.49,0.56,0.16}{#1}}
\newcommand{\BaseNTok}[1]{\textcolor[rgb]{0.25,0.63,0.44}{#1}}
\newcommand{\BuiltInTok}[1]{#1}
\newcommand{\CharTok}[1]{\textcolor[rgb]{0.25,0.44,0.63}{#1}}
\newcommand{\CommentTok}[1]{\textcolor[rgb]{0.38,0.63,0.69}{\textit{#1}}}
\newcommand{\CommentVarTok}[1]{\textcolor[rgb]{0.38,0.63,0.69}{\textbf{\textit{#1}}}}
\newcommand{\ConstantTok}[1]{\textcolor[rgb]{0.53,0.00,0.00}{#1}}
\newcommand{\ControlFlowTok}[1]{\textcolor[rgb]{0.00,0.44,0.13}{\textbf{#1}}}
\newcommand{\DataTypeTok}[1]{\textcolor[rgb]{0.56,0.13,0.00}{#1}}
\newcommand{\DecValTok}[1]{\textcolor[rgb]{0.25,0.63,0.44}{#1}}
\newcommand{\DocumentationTok}[1]{\textcolor[rgb]{0.73,0.13,0.13}{\textit{#1}}}
\newcommand{\ErrorTok}[1]{\textcolor[rgb]{1.00,0.00,0.00}{\textbf{#1}}}
\newcommand{\ExtensionTok}[1]{#1}
\newcommand{\FloatTok}[1]{\textcolor[rgb]{0.25,0.63,0.44}{#1}}
\newcommand{\FunctionTok}[1]{\textcolor[rgb]{0.02,0.16,0.49}{#1}}
\newcommand{\ImportTok}[1]{#1}
\newcommand{\InformationTok}[1]{\textcolor[rgb]{0.38,0.63,0.69}{\textbf{\textit{#1}}}}
\newcommand{\KeywordTok}[1]{\textcolor[rgb]{0.00,0.44,0.13}{\textbf{#1}}}
\newcommand{\NormalTok}[1]{#1}
\newcommand{\OperatorTok}[1]{\textcolor[rgb]{0.40,0.40,0.40}{#1}}
\newcommand{\OtherTok}[1]{\textcolor[rgb]{0.00,0.44,0.13}{#1}}
\newcommand{\PreprocessorTok}[1]{\textcolor[rgb]{0.74,0.48,0.00}{#1}}
\newcommand{\RegionMarkerTok}[1]{#1}
\newcommand{\SpecialCharTok}[1]{\textcolor[rgb]{0.25,0.44,0.63}{#1}}
\newcommand{\SpecialStringTok}[1]{\textcolor[rgb]{0.73,0.40,0.53}{#1}}
\newcommand{\StringTok}[1]{\textcolor[rgb]{0.25,0.44,0.63}{#1}}
\newcommand{\VariableTok}[1]{\textcolor[rgb]{0.10,0.09,0.49}{#1}}
\newcommand{\VerbatimStringTok}[1]{\textcolor[rgb]{0.25,0.44,0.63}{#1}}
\newcommand{\WarningTok}[1]{\textcolor[rgb]{0.38,0.63,0.69}{\textbf{\textit{#1}}}}

% longtable

% multiplecol
\usepackage{multicol}

% strikeout
\usepackage[normalem]{ulem}

% morefloats
\usepackage{morefloats}


% tightlist macro required by pandoc >= 1.14
\providecommand{\tightlist}{%
  \setlength{\itemsep}{0pt}\setlength{\parskip}{0pt}}

% title / author / date
\title{numpy continued}
\author{Ben Bolker}
\date{07 November 2019}


\begin{document}

\maketitle




\hypertarget{operations-along-axes}{%
\subsection{operations along axes}\label{operations-along-axes}}

\begin{itemize}
\tightlist
\item
  array axes are numbered

  \begin{itemize}
  \tightlist
  \item
    0 = rows
  \item
    1 = columns
  \item
    2 = ``slices''
  \end{itemize}
\end{itemize}

From
\href{https://www.sharpsightlabs.com/blog/numpy-axes-explained/}{here}:

\begin{quote}
When you use the NumPy sum function with the axis parameter, the axis
that you specify is the axis that gets collapsed.
\end{quote}

\hypertarget{examples}{%
\subsection{examples}\label{examples}}

\begin{Shaded}
\begin{Highlighting}[]
\ImportTok{import}\NormalTok{ numpy }\ImportTok{as}\NormalTok{ np}
\NormalTok{a }\OperatorTok{=}\NormalTok{ np.arange(}\DecValTok{25}\NormalTok{).reshape((}\DecValTok{5}\NormalTok{,}\DecValTok{5}\NormalTok{))}
\BuiltInTok{print}\NormalTok{(a)}
\end{Highlighting}
\end{Shaded}

\begin{verbatim}
## [[ 0  1  2  3  4]
##  [ 5  6  7  8  9]
##  [10 11 12 13 14]
##  [15 16 17 18 19]
##  [20 21 22 23 24]]
\end{verbatim}

\begin{Shaded}
\begin{Highlighting}[]
\BuiltInTok{print}\NormalTok{(a.}\BuiltInTok{sum}\NormalTok{())       }\CommentTok{## axis=None, collapse everything}
\end{Highlighting}
\end{Shaded}

\begin{verbatim}
## 300
\end{verbatim}

\begin{Shaded}
\begin{Highlighting}[]
\BuiltInTok{print}\NormalTok{(a.}\BuiltInTok{sum}\NormalTok{(axis}\OperatorTok{=}\DecValTok{0}\NormalTok{)) }\CommentTok{## sum *across* rows, collapse rows}
\end{Highlighting}
\end{Shaded}

\begin{verbatim}
## [50 55 60 65 70]
\end{verbatim}

\begin{Shaded}
\begin{Highlighting}[]
\BuiltInTok{print}\NormalTok{(a.}\BuiltInTok{sum}\NormalTok{(axis}\OperatorTok{=}\DecValTok{1}\NormalTok{)) }\CommentTok{## sum *across* columns, collapse columns}
\end{Highlighting}
\end{Shaded}

\begin{verbatim}
## [ 10  35  60  85 110]
\end{verbatim}

\hypertarget{try-a-3-d-array}{%
\subsection{try a 3-D array}\label{try-a-3-d-array}}

\begin{Shaded}
\begin{Highlighting}[]
\NormalTok{b }\OperatorTok{=}\NormalTok{ np.arange(}\DecValTok{24}\NormalTok{).reshape((}\DecValTok{2}\NormalTok{,}\DecValTok{3}\NormalTok{,}\DecValTok{4}\NormalTok{))}
\BuiltInTok{print}\NormalTok{(b)  }\CommentTok{## 2 slices, 3 rows, 4 columns}
\end{Highlighting}
\end{Shaded}

\begin{verbatim}
## [[[ 0  1  2  3]
##   [ 4  5  6  7]
##   [ 8  9 10 11]]
## 
##  [[12 13 14 15]
##   [16 17 18 19]
##   [20 21 22 23]]]
\end{verbatim}

\begin{Shaded}
\begin{Highlighting}[]
\BuiltInTok{print}\NormalTok{(b.}\BuiltInTok{sum}\NormalTok{())}
\end{Highlighting}
\end{Shaded}

\begin{verbatim}
## 276
\end{verbatim}

\begin{Shaded}
\begin{Highlighting}[]
\BuiltInTok{print}\NormalTok{(b.}\BuiltInTok{sum}\NormalTok{(axis}\OperatorTok{=}\DecValTok{0}\NormalTok{))}
\end{Highlighting}
\end{Shaded}

\begin{verbatim}
## [[12 14 16 18]
##  [20 22 24 26]
##  [28 30 32 34]]
\end{verbatim}

\begin{Shaded}
\begin{Highlighting}[]
\BuiltInTok{print}\NormalTok{(b.}\BuiltInTok{sum}\NormalTok{(axis}\OperatorTok{=}\DecValTok{1}\NormalTok{))}
\end{Highlighting}
\end{Shaded}

\begin{verbatim}
## [[12 15 18 21]
##  [48 51 54 57]]
\end{verbatim}

\begin{Shaded}
\begin{Highlighting}[]
\BuiltInTok{print}\NormalTok{(b.}\BuiltInTok{sum}\NormalTok{(axis}\OperatorTok{=}\DecValTok{2}\NormalTok{))}
\end{Highlighting}
\end{Shaded}

\begin{verbatim}
## [[ 6 22 38]
##  [54 70 86]]
\end{verbatim}

\hypertarget{broadcasting}{%
\subsection{broadcasting}\label{broadcasting}}

\begin{itemize}
\tightlist
\item
  \textbf{broadcasting} means matching up dimensions when doing
  operations on two non-matching arrays.
\item
  errors may be thrown if arrays do not match in size, e.g.
\end{itemize}

\begin{Shaded}
\begin{Highlighting}[]
\NormalTok{np.array([}\DecValTok{1}\NormalTok{, }\DecValTok{2}\NormalTok{, }\DecValTok{3}\NormalTok{]) }\OperatorTok{+}\NormalTok{ np.array([}\DecValTok{4}\NormalTok{, }\DecValTok{5}\NormalTok{])}
\CommentTok{## ValueError: operands could not be broadcast together with shapes (3,) (2,)}
\end{Highlighting}
\end{Shaded}

\begin{itemize}
\tightlist
\item
  arrays that do not match in the number of \textbf{dimensions} will be
  broadcast (to perform mathematical operations)
\item
  the smaller array will be repeated as necessary
\end{itemize}

\begin{Shaded}
\begin{Highlighting}[]
\NormalTok{a }\OperatorTok{=}\NormalTok{ np.array([[}\DecValTok{1}\NormalTok{, }\DecValTok{2}\NormalTok{], [}\DecValTok{3}\NormalTok{, }\DecValTok{4}\NormalTok{], [}\DecValTok{5}\NormalTok{, }\DecValTok{6}\NormalTok{]], }\BuiltInTok{float}\NormalTok{)}
\NormalTok{b }\OperatorTok{=}\NormalTok{ np.array([}\OperatorTok{-}\DecValTok{1}\NormalTok{, }\DecValTok{3}\NormalTok{], }\BuiltInTok{float}\NormalTok{)}
\BuiltInTok{print}\NormalTok{(a }\OperatorTok{+}\NormalTok{ b)}
\end{Highlighting}
\end{Shaded}

\begin{verbatim}
## [[0. 5.]
##  [2. 7.]
##  [4. 9.]]
\end{verbatim}

\begin{center}\rule{0.5\linewidth}{\linethickness}\end{center}

\begin{itemize}
\tightlist
\item
  sometimes it doesn't work
\end{itemize}

\begin{Shaded}
\begin{Highlighting}[]
\NormalTok{c }\OperatorTok{=}\NormalTok{ np.arange(}\DecValTok{3}\NormalTok{)}
\end{Highlighting}
\end{Shaded}

\begin{Shaded}
\begin{Highlighting}[]
\NormalTok{a }\OperatorTok{+}\NormalTok{ c}
\CommentTok{## ValueError: operands could not be broadcast together with shapes (3,2) (3,)}
\end{Highlighting}
\end{Shaded}

\begin{itemize}
\tightlist
\item
  you could reshape it:
\end{itemize}

\begin{Shaded}
\begin{Highlighting}[]
\NormalTok{a }\OperatorTok{+}\NormalTok{ c.reshape(}\DecValTok{3}\NormalTok{,}\DecValTok{1}\NormalTok{)}
\end{Highlighting}
\end{Shaded}

\begin{verbatim}
## array([[1., 2.],
##        [4., 5.],
##        [7., 8.]])
\end{verbatim}

\begin{itemize}
\tightlist
\item
  or use slicing with \texttt{np.newaxis}
\end{itemize}

\begin{Shaded}
\begin{Highlighting}[]
\BuiltInTok{print}\NormalTok{(c)}
\end{Highlighting}
\end{Shaded}

\begin{verbatim}
## [0 1 2]
\end{verbatim}

\begin{Shaded}
\begin{Highlighting}[]
\BuiltInTok{print}\NormalTok{(c[:])}
\end{Highlighting}
\end{Shaded}

\begin{verbatim}
## [0 1 2]
\end{verbatim}

\begin{Shaded}
\begin{Highlighting}[]
\BuiltInTok{print}\NormalTok{(c[np.newaxis,:])}
\end{Highlighting}
\end{Shaded}

\begin{verbatim}
## [[0 1 2]]
\end{verbatim}

\begin{Shaded}
\begin{Highlighting}[]
\BuiltInTok{print}\NormalTok{(c[:,np.newaxis])}
\end{Highlighting}
\end{Shaded}

\begin{verbatim}
## [[0]
##  [1]
##  [2]]
\end{verbatim}

\begin{Shaded}
\begin{Highlighting}[]
\NormalTok{a }\OperatorTok{+}\NormalTok{ c[:,np.newaxis]}
\end{Highlighting}
\end{Shaded}

\begin{verbatim}
## array([[1., 2.],
##        [4., 5.],
##        [7., 8.]])
\end{verbatim}

\begin{itemize}
\tightlist
\item
  think of \texttt{np.newaxis} as adding a new, \emph{length-one}
  dimension
\end{itemize}

\hypertarget{matrix-and-vector-math}{%
\subsection{matrix and vector math}\label{matrix-and-vector-math}}

\begin{itemize}
\tightlist
\item
  dot products: use the \texttt{np.dot()} function
\end{itemize}

\begin{Shaded}
\begin{Highlighting}[]
\NormalTok{c }\OperatorTok{=}\NormalTok{ np.arange(}\DecValTok{4}\NormalTok{,}\DecValTok{7}\NormalTok{)}
\NormalTok{d }\OperatorTok{=}\NormalTok{ np.arange(}\OperatorTok{-}\DecValTok{1}\NormalTok{,}\OperatorTok{-}\DecValTok{4}\NormalTok{,}\OperatorTok{-}\DecValTok{1}\NormalTok{)}
\BuiltInTok{print}\NormalTok{(np.dot(c,d))}
\end{Highlighting}
\end{Shaded}

\begin{verbatim}
## -32
\end{verbatim}

\begin{itemize}
\tightlist
\item
  \texttt{.dot()} also works for matrix multiplication
\item
  here we multiply \texttt{a} = (3x2) x \texttt{e} = (2x4) to get a 3x4
  matrix
\end{itemize}

\begin{Shaded}
\begin{Highlighting}[]
\NormalTok{e }\OperatorTok{=}\NormalTok{ np.array([[}\DecValTok{1}\NormalTok{, }\DecValTok{0}\NormalTok{, }\DecValTok{2}\NormalTok{, }\DecValTok{-1}\NormalTok{], [}\DecValTok{0}\NormalTok{, }\DecValTok{1}\NormalTok{, }\DecValTok{2}\NormalTok{, }\DecValTok{-3}\NormalTok{]])}
\BuiltInTok{print}\NormalTok{(np.dot(a,e))}
\end{Highlighting}
\end{Shaded}

\begin{verbatim}
## [[  1.   2.   6.  -7.]
##  [  3.   4.  14. -15.]
##  [  5.   6.  22. -23.]]
\end{verbatim}

\hypertarget{more-matrix-math}{%
\subsection{more matrix math}\label{more-matrix-math}}

\begin{itemize}
\tightlist
\item
  get transposes with \texttt{a.T} or \texttt{np.transpose(a)}
\item
  the \texttt{linalg} submodule does non-trivial linear algebra:
  determinants, inverses, eigenvalues and eigenvectors
\end{itemize}

\begin{Shaded}
\begin{Highlighting}[]
\NormalTok{a }\OperatorTok{=}\NormalTok{ np.array([[}\DecValTok{4}\NormalTok{, }\DecValTok{2}\NormalTok{, }\DecValTok{0}\NormalTok{], [}\DecValTok{9}\NormalTok{, }\DecValTok{3}\NormalTok{, }\DecValTok{7}\NormalTok{], [}\DecValTok{1}\NormalTok{, }\DecValTok{2}\NormalTok{, }\DecValTok{1}\NormalTok{]])}
\BuiltInTok{print}\NormalTok{(np.linalg.det(a))}
\end{Highlighting}
\end{Shaded}

\begin{verbatim}
## -48.00000000000003
\end{verbatim}

\begin{Shaded}
\begin{Highlighting}[]
\ImportTok{import}\NormalTok{ numpy.linalg }\ImportTok{as}\NormalTok{ npl  }\CommentTok{## shortcut}
\NormalTok{npl.det(a)}
\end{Highlighting}
\end{Shaded}

\begin{verbatim}
## -48.00000000000003
\end{verbatim}

\hypertarget{inverses}{%
\subsection{inverses}\label{inverses}}

\begin{Shaded}
\begin{Highlighting}[]
\BuiltInTok{print}\NormalTok{(npl.inv(a))}
\end{Highlighting}
\end{Shaded}

\begin{verbatim}
## [[ 0.22916667  0.04166667 -0.29166667]
##  [ 0.04166667 -0.08333333  0.58333333]
##  [-0.3125      0.125       0.125     ]]
\end{verbatim}

\begin{Shaded}
\begin{Highlighting}[]
\NormalTok{m }\OperatorTok{=}\NormalTok{ np.dot(a,npl.inv(a))}
\BuiltInTok{print}\NormalTok{(m)}
\end{Highlighting}
\end{Shaded}

\begin{verbatim}
## [[1.00000000e+00 5.55111512e-17 0.00000000e+00]
##  [0.00000000e+00 1.00000000e+00 2.22044605e-16]
##  [0.00000000e+00 1.38777878e-17 1.00000000e+00]]
\end{verbatim}

\begin{Shaded}
\begin{Highlighting}[]
\BuiltInTok{print}\NormalTok{(m.}\BuiltInTok{round}\NormalTok{())}
\end{Highlighting}
\end{Shaded}

\begin{verbatim}
## [[1. 0. 0.]
##  [0. 1. 0.]
##  [0. 0. 1.]]
\end{verbatim}

\hypertarget{eigenstuff}{%
\subsection{eigenstuff}\label{eigenstuff}}

\begin{Shaded}
\begin{Highlighting}[]
\NormalTok{vals, vecs }\OperatorTok{=}\NormalTok{ npl.eig(a) }\CommentTok{## unpack}
\BuiltInTok{print}\NormalTok{(vals)}
\end{Highlighting}
\end{Shaded}

\begin{verbatim}
## [ 8.85591316  1.9391628  -2.79507597]
\end{verbatim}

\begin{Shaded}
\begin{Highlighting}[]
\BuiltInTok{print}\NormalTok{(vecs)}
\end{Highlighting}
\end{Shaded}

\begin{verbatim}
## [[-0.3663565  -0.54736745  0.25928158]
##  [-0.88949768  0.5640176  -0.88091903]
##  [-0.27308752  0.61828231  0.39592263]]
\end{verbatim}

\hypertarget{testing-eigenstuff}{%
\subsection{testing eigenstuff}\label{testing-eigenstuff}}

We expect \(A e_0 = \lambda_a e_0\). Does it work?

\begin{Shaded}
\begin{Highlighting}[]
\NormalTok{e0 }\OperatorTok{=}\NormalTok{ vecs[:,}\DecValTok{0}\NormalTok{]}
\BuiltInTok{print}\NormalTok{(np.isclose(np.dot(a,e0),vals[}\DecValTok{0}\NormalTok{]}\OperatorTok{*}\NormalTok{e0))}
\end{Highlighting}
\end{Shaded}

\begin{verbatim}
## [ True  True  True]
\end{verbatim}

\hypertarget{array-iteration}{%
\subsection{array iteration}\label{array-iteration}}

\begin{itemize}
\tightlist
\item
  arrays can be iterated over in a similar way to lists
\item
  the statement \texttt{for\ x\ in\ a:} will iterate over the
  \emph{first} (0) axis of \texttt{a}
\end{itemize}

\begin{Shaded}
\begin{Highlighting}[]
\NormalTok{c }\OperatorTok{=}\NormalTok{ np.arange(}\DecValTok{2}\NormalTok{, }\DecValTok{10}\NormalTok{, }\DecValTok{3}\NormalTok{, dtype}\OperatorTok{=}\BuiltInTok{float}\NormalTok{)}
\ControlFlowTok{for}\NormalTok{ x }\KeywordTok{in}\NormalTok{ c:}
   \BuiltInTok{print}\NormalTok{(x)}
\end{Highlighting}
\end{Shaded}

\begin{Shaded}
\begin{Highlighting}[]
\ControlFlowTok{for}\NormalTok{ x }\KeywordTok{in}\NormalTok{ a:}
    \BuiltInTok{print}\NormalTok{(a)}
\end{Highlighting}
\end{Shaded}

\begin{verbatim}
## [[4 2 0]
##  [9 3 7]
##  [1 2 1]]
## [[4 2 0]
##  [9 3 7]
##  [1 2 1]]
## [[4 2 0]
##  [9 3 7]
##  [1 2 1]]
\end{verbatim}

\hypertarget{logical-arrays}{%
\subsection{logical arrays}\label{logical-arrays}}

\begin{itemize}
\tightlist
\item
  vectorized logical comparisons
\item
  e.g.~\texttt{a\textgreater{}0} gives an array of \texttt{bool}
\end{itemize}

\begin{Shaded}
\begin{Highlighting}[]
\NormalTok{a }\OperatorTok{=}\NormalTok{ np.array([}\DecValTok{2}\NormalTok{, }\DecValTok{4}\NormalTok{, }\DecValTok{6}\NormalTok{], }\BuiltInTok{float}\NormalTok{)}
\NormalTok{b }\OperatorTok{=}\NormalTok{ np.array([}\DecValTok{4}\NormalTok{, }\DecValTok{2}\NormalTok{, }\DecValTok{6}\NormalTok{], }\BuiltInTok{float}\NormalTok{)}
\NormalTok{result1 }\OperatorTok{=}\NormalTok{ (a }\OperatorTok{>}\NormalTok{ b)}
\NormalTok{result2 }\OperatorTok{=}\NormalTok{ (a }\OperatorTok{==}\NormalTok{ b)}
\BuiltInTok{print}\NormalTok{(result1, result2)}
\end{Highlighting}
\end{Shaded}

\begin{verbatim}
## [False  True False] [False False  True]
\end{verbatim}

\hypertarget{more-examples}{%
\subsection{more examples}\label{more-examples}}

\begin{Shaded}
\begin{Highlighting}[]
\CommentTok{## compare with scalar}
\BuiltInTok{print}\NormalTok{(a}\OperatorTok{>}\DecValTok{3}\NormalTok{)}
\end{Highlighting}
\end{Shaded}

\begin{verbatim}
## [False  True  True]
\end{verbatim}

\begin{itemize}
\tightlist
\item
  \texttt{any} and \texttt{all} and logical expressions work:
\end{itemize}

\begin{Shaded}
\begin{Highlighting}[]
\NormalTok{c }\OperatorTok{=}\NormalTok{ np.array([}\VariableTok{True}\NormalTok{, }\VariableTok{False}\NormalTok{, }\VariableTok{False}\NormalTok{])}
\NormalTok{d }\OperatorTok{=}\NormalTok{ np.array([}\VariableTok{False}\NormalTok{, }\VariableTok{False}\NormalTok{, }\VariableTok{True}\NormalTok{])}
\BuiltInTok{print}\NormalTok{(}\BuiltInTok{any}\NormalTok{(c), }\BuiltInTok{all}\NormalTok{(c))}
\end{Highlighting}
\end{Shaded}

\begin{verbatim}
## True False
\end{verbatim}

\begin{Shaded}
\begin{Highlighting}[]
\BuiltInTok{print}\NormalTok{(np.logical_and(c,d))}
\end{Highlighting}
\end{Shaded}

\begin{verbatim}
## [False False False]
\end{verbatim}

\begin{Shaded}
\begin{Highlighting}[]
\BuiltInTok{print}\NormalTok{(np.logical_or(a}\OperatorTok{>}\DecValTok{4}\NormalTok{,a}\OperatorTok{<}\DecValTok{3}\NormalTok{))}
\end{Highlighting}
\end{Shaded}

\begin{verbatim}
## [ True False  True]
\end{verbatim}

\hypertarget{selecting-based-on-logical-values}{%
\subsection{selecting based on logical
values}\label{selecting-based-on-logical-values}}

\begin{Shaded}
\begin{Highlighting}[]
\BuiltInTok{print}\NormalTok{(a[a }\OperatorTok{>=} \DecValTok{6}\NormalTok{])}
\end{Highlighting}
\end{Shaded}

\begin{verbatim}
## [6.]
\end{verbatim}

\begin{Shaded}
\begin{Highlighting}[]
\NormalTok{sel }\OperatorTok{=}\NormalTok{ np.logical_and(a}\OperatorTok{>}\DecValTok{5}\NormalTok{, a}\OperatorTok{<}\DecValTok{9}\NormalTok{)}
\BuiltInTok{print}\NormalTok{(a[sel])}
\end{Highlighting}
\end{Shaded}

\begin{verbatim}
## [6.]
\end{verbatim}

Set all elements of \texttt{a} that are \textgreater4 to 0:

\begin{Shaded}
\begin{Highlighting}[]
\NormalTok{a[a}\OperatorTok{>}\DecValTok{4}\NormalTok{] }\OperatorTok{=} \DecValTok{0}
\BuiltInTok{print}\NormalTok{(a)}
\end{Highlighting}
\end{Shaded}

\begin{verbatim}
## [2. 4. 0.]
\end{verbatim}

\hypertarget{examples-1}{%
\subsection{examples}\label{examples-1}}

Many examples
\href{http://www.labri.fr/perso/nrougier/teaching/numpy.100/index.html}{here}
(or
\href{http://mybinder.org/repo/rougier/numpy-100/notebooks/100_Numpy_exercises.ipynb}{here}),
e.g.

-calculate the mean of the squares of the natural numbers up to 7 -
create a 5 x 5 array with row values ranging from 0 to 1 by 0.2 - create
a 3 x 7 array containing the values 0 to 20 and a 7 x 3 array containing
the values 0 to 20 and matrix-multiply them: the result should be

\begin{verbatim}
## [[ 273  294  315]
##  [ 714  784  854]
##  [1155 1274 1393]]
\end{verbatim}

\hypertarget{gamblers-ruin-revisited}{%
\subsection{gambler's ruin revisited}\label{gamblers-ruin-revisited}}

A slightly more compact version of the ``gambler's ruin'' code (i.e., a
Markov chain starting at a particular value and going up or down by one
unit at each step with a probability of \(p\) or \(1-p\) respectively.

\begin{Shaded}
\begin{Highlighting}[]
\ImportTok{import}\NormalTok{ numpy }\ImportTok{as}\NormalTok{ np}
\ImportTok{import}\NormalTok{ numpy.random }\ImportTok{as}\NormalTok{ npr}
\KeywordTok{def}\NormalTok{ gamblers_ruin(start}\OperatorTok{=}\DecValTok{10}\NormalTok{,}\BuiltInTok{max}\OperatorTok{=}\DecValTok{50}\NormalTok{,prob}\OperatorTok{=}\FloatTok{0.5}\NormalTok{):}
    \CommentTok{## iterate until you get to zero or max}
    \CommentTok{## return tuple: (0 = lost, 1 = won,}
    \CommentTok{##   [number of steps]}
\NormalTok{    i }\OperatorTok{=} \DecValTok{0}
\NormalTok{    x }\OperatorTok{=}\NormalTok{ start}
    \ControlFlowTok{while}\NormalTok{ x}\OperatorTok{>}\DecValTok{0} \KeywordTok{and}\NormalTok{ x}\OperatorTok{<}\BuiltInTok{max}\NormalTok{:}
\NormalTok{        r }\OperatorTok{=}\NormalTok{ npr.uniform()}
\NormalTok{        x }\OperatorTok{+=}\NormalTok{ np.sign(prob}\OperatorTok{-}\NormalTok{r) }\CommentTok{## +/- 1}
\NormalTok{    result }\OperatorTok{=} \BuiltInTok{int}\NormalTok{(x}\OperatorTok{>}\DecValTok{0}\NormalTok{)}
    \ControlFlowTok{return}\NormalTok{(np.array((result, i)))}
\end{Highlighting}
\end{Shaded}

Simulate 1000 games:

\begin{Shaded}
\begin{Highlighting}[]
\NormalTok{sim }\OperatorTok{=}\NormalTok{ np.zeros((}\DecValTok{1000}\NormalTok{,}\DecValTok{2}\NormalTok{))}
\ControlFlowTok{for}\NormalTok{ i }\KeywordTok{in} \BuiltInTok{range}\NormalTok{(}\DecValTok{1000}\NormalTok{):}
\NormalTok{    sim[i,:] }\OperatorTok{=}\NormalTok{ gamblers_ruin()}
\end{Highlighting}
\end{Shaded}

Evaluate results:

\begin{Shaded}
\begin{Highlighting}[]
\NormalTok{sim[:,}\DecValTok{0}\NormalTok{].mean()  }\CommentTok{## prob of winning}
\end{Highlighting}
\end{Shaded}

\begin{verbatim}
## 0.203
\end{verbatim}

\begin{Shaded}
\begin{Highlighting}[]
\NormalTok{sim[:,}\DecValTok{1}\NormalTok{].}\BuiltInTok{max}\NormalTok{()   }\CommentTok{## max number of steps}
\end{Highlighting}
\end{Shaded}

\begin{verbatim}
## 0.0
\end{verbatim}

\begin{Shaded}
\begin{Highlighting}[]
\NormalTok{sim[:,}\DecValTok{1}\NormalTok{].}\BuiltInTok{min}\NormalTok{()   }\CommentTok{## min number of steps}
\end{Highlighting}
\end{Shaded}

\begin{verbatim}
## 0.0
\end{verbatim}

\begin{Shaded}
\begin{Highlighting}[]
\NormalTok{lost }\OperatorTok{=}\NormalTok{ sim[:,}\DecValTok{0}\NormalTok{]}\OperatorTok{==}\DecValTok{0}
\NormalTok{sim[lost,}\DecValTok{1}\NormalTok{].mean()}
\end{Highlighting}
\end{Shaded}

\begin{verbatim}
## 0.0
\end{verbatim}

\begin{Shaded}
\begin{Highlighting}[]
\NormalTok{sim[np.logical_not(lost),}\DecValTok{1}\NormalTok{].mean()}
\end{Highlighting}
\end{Shaded}

\begin{verbatim}
## 0.0
\end{verbatim}

We can try this for different starting values, upper bounds,
probabilities of winning, etc.: see
e.g.~\href{http://www.columbia.edu/~ks20/FE-Notes/4700-07-Notes-GR.pdf}{here}
for the derivation of the analytical solution:

\[
P_i = \begin{cases}
\frac{1- \left( \frac{q}{p} \right)^i}{1- \left( \frac{q}{p}\right)^N} \quad , 
& \textrm{if } p \neq q \\
\frac{i}{N} \quad , & \textrm{if } p=q=0.5
\end{cases}
\] where \(i\)=starting value; \(p\)=winning probability; \(q=1-p\);
\(N\)=upper bound

\hypertarget{numerics}{%
\section{numerics}\label{numerics}}

\hypertarget{section}{%
\subsection{}\label{section}}

\begin{itemize}
\tightlist
\item
  In Python, numbers are stored as binary digits (bits).
\item
  If \texttt{n} bits are available to store a \textbf{signed} integer,
  we use one bit to indicate the sign; this gives room to store
  \textbf{signed} values between \(−2^{n−1}\) and \(2^{n−1} − 1\)
\item
  So, 64 bits can be used to store any integer between
  -9223372036854775808 and 9223372036854775807 (since \(2^63 - 1\) =
  9223372036854775807). Fortunately, base Python automatically uses as
  many bits as necessary to store arbitrary-length integers
\end{itemize}

\begin{Shaded}
\begin{Highlighting}[]
\NormalTok{a }\OperatorTok{=} \DecValTok{2} \OperatorTok{**} \DecValTok{63} \OperatorTok{-} \DecValTok{1}
\NormalTok{b }\OperatorTok{=}\NormalTok{ a }\OperatorTok{*} \DecValTok{100000}
\BuiltInTok{print}\NormalTok{(}\StringTok{"a = "}\NormalTok{,a, }\StringTok{", b = "}\NormalTok{,b)}
\end{Highlighting}
\end{Shaded}

\begin{verbatim}
## a =  9223372036854775807 , b =  922337203685477580700000
\end{verbatim}

\begin{center}\rule{0.5\linewidth}{\linethickness}\end{center}

\begin{itemize}
\tightlist
\item
  In other languages, and with \texttt{numpy} arrays, you need to be
  careful!
\item
  The default type for integers within numpy is int32 or int64 but this
  might depend on your hardware/operating system
\end{itemize}

\begin{Shaded}
\begin{Highlighting}[]
\NormalTok{a }\OperatorTok{=}\NormalTok{ np.array([}\DecValTok{2} \OperatorTok{**} \DecValTok{63} \OperatorTok{-} \DecValTok{1}\NormalTok{])}
\NormalTok{b }\OperatorTok{=}\NormalTok{ np.array([}\DecValTok{2} \OperatorTok{**} \DecValTok{31} \OperatorTok{-} \DecValTok{1}\NormalTok{])}
\BuiltInTok{print}\NormalTok{(a.dtype, b.dtype)}
\end{Highlighting}
\end{Shaded}

\begin{verbatim}
## int64 int64
\end{verbatim}

\begin{itemize}
\tightlist
\item
  If you're not using huge integers (i.e.~\textgreater{} \(2^63-1\)),
  you don't need to worry
\item
  You have
  \href{https://docs.scipy.org/doc/numpy/user/basics.types.html}{lots of
  choices}, including

  \begin{itemize}
  \tightlist
  \item
    int8, int16, int32, int64
  \item
    \textbf{unsigned} values: uint8, uint16, uint32 uint64
  \end{itemize}
\end{itemize}

\begin{center}\rule{0.5\linewidth}{\linethickness}\end{center}

\begin{itemize}
\tightlist
\item
  for small sizes, or huge numbers, you can get \textbf{overflow}
\end{itemize}

\begin{Shaded}
\begin{Highlighting}[]
\NormalTok{a }\OperatorTok{=}\NormalTok{ np.array([}\DecValTok{1}\NormalTok{], dtype}\OperatorTok{=}\StringTok{"int8"}\NormalTok{) }\CommentTok{## 8-bit integer (-127 to 128)}
\BuiltInTok{print}\NormalTok{(}\BuiltInTok{bin}\NormalTok{(a[}\DecValTok{0}\NormalTok{]))}
\end{Highlighting}
\end{Shaded}

\begin{verbatim}
## 0b1
\end{verbatim}

\begin{Shaded}
\begin{Highlighting}[]
\NormalTok{a[}\DecValTok{0}\NormalTok{] }\OperatorTok{=} \DecValTok{127}
\BuiltInTok{print}\NormalTok{(}\BuiltInTok{bin}\NormalTok{(a[}\DecValTok{0}\NormalTok{]))}
\end{Highlighting}
\end{Shaded}

\begin{verbatim}
## 0b1111111
\end{verbatim}

\begin{Shaded}
\begin{Highlighting}[]
\NormalTok{a[}\DecValTok{0}\NormalTok{] }\OperatorTok{+=} \DecValTok{1}
\BuiltInTok{print}\NormalTok{(}\BuiltInTok{bin}\NormalTok{(a[}\DecValTok{0}\NormalTok{]))}
\end{Highlighting}
\end{Shaded}

\begin{verbatim}
## -0b10000000
\end{verbatim}

\begin{Shaded}
\begin{Highlighting}[]
\BuiltInTok{print}\NormalTok{(a)}
\end{Highlighting}
\end{Shaded}

\begin{verbatim}
## [-128]
\end{verbatim}

\textbf{be careful} (\href{https://xkcd.com/571/}{obligatory xkcd})

\hypertarget{floats}{%
\subsection{floats}\label{floats}}

\begin{itemize}
\tightlist
\item
  Floating point numbers are represented in computer hardware as
  \textbf{binary fractions} plus
\item
  Many decimal fractions cannot be represented exactly as binary
  fractions
\item
  This can lead to unexpected or suprising results.
\end{itemize}

\begin{Shaded}
\begin{Highlighting}[]
\BuiltInTok{print}\NormalTok{(}\StringTok{"2/3 = "}\NormalTok{,}\DecValTok{2} \OperatorTok{/} \DecValTok{3}\NormalTok{,}\StringTok{" 2/3 + 1 ="}\NormalTok{,}\DecValTok{2}\OperatorTok{/}\DecValTok{3} \OperatorTok{+} \DecValTok{1}\NormalTok{, }\StringTok{"}\CharTok{\textbackslash{}n}\StringTok{"}\NormalTok{,}
\StringTok{" 5/3 ="}\NormalTok{, }\DecValTok{5}\OperatorTok{/}\DecValTok{3}\NormalTok{)}
\end{Highlighting}
\end{Shaded}

\begin{verbatim}
## 2/3 =  0.6666666666666666  2/3 + 1 = 1.6666666666666665 
##   5/3 = 1.6666666666666667
\end{verbatim}

\begin{Shaded}
\begin{Highlighting}[]
\BuiltInTok{print}\NormalTok{(}\StringTok{"1.13 - 1.1 ="}\NormalTok{, }\FloatTok{1.13} \OperatorTok{-} \FloatTok{1.1}\NormalTok{, }\StringTok{"}\CharTok{\textbackslash{}n}\StringTok{3.13 - 1.1 ="}\NormalTok{, }\FloatTok{3.13} \OperatorTok{-} \FloatTok{1.1}\NormalTok{)}
\end{Highlighting}
\end{Shaded}

\begin{verbatim}
## 1.13 - 1.1 = 0.029999999999999805 
## 3.13 - 1.1 = 2.03
\end{verbatim}

\begin{Shaded}
\begin{Highlighting}[]
\BuiltInTok{print}\NormalTok{(}\StringTok{"1+1e-15 ="}\NormalTok{,}\DecValTok{1}\FloatTok{+1e-15}\NormalTok{, }\StringTok{"}\CharTok{\textbackslash{}n}\StringTok{1+1e-16 ="}\NormalTok{,}\DecValTok{1}\FloatTok{+1e-16}\NormalTok{)}
\end{Highlighting}
\end{Shaded}

\begin{verbatim}
## 1+1e-15 = 1.000000000000001 
## 1+1e-16 = 1.0
\end{verbatim}

\begin{Shaded}
\begin{Highlighting}[]
\NormalTok{a }\OperatorTok{=} \BuiltInTok{float}\NormalTok{(}\DecValTok{1234567890123456}\NormalTok{)}
\BuiltInTok{print}\NormalTok{(}\StringTok{"a="}\NormalTok{,a,}\StringTok{"}\CharTok{\textbackslash{}n}\StringTok{a*10="}\NormalTok{,a}\OperatorTok{*}\DecValTok{10}\NormalTok{)}
\end{Highlighting}
\end{Shaded}

\begin{verbatim}
## a= 1234567890123456.0 
## a*10= 1.234567890123456e+16
\end{verbatim}

\begin{center}\rule{0.5\linewidth}{\linethickness}\end{center}

\begin{itemize}
\tightlist
\item
  None of these results are errors: they are an inevitable outcome of
  finite precision
\item
  Small differences \textbf{might} not matter, but they can accumulate,
  and
\end{itemize}

\begin{Shaded}
\begin{Highlighting}[]
\NormalTok{sqrt2 }\OperatorTok{=}\NormalTok{ np.sqrt(}\DecValTok{2}\NormalTok{)}
\NormalTok{sqrt2}\OperatorTok{**}\DecValTok{2}\OperatorTok{==}\FloatTok{2.0}
\end{Highlighting}
\end{Shaded}

\begin{verbatim}
## False
\end{verbatim}

\begin{Shaded}
\begin{Highlighting}[]
\NormalTok{np.isclose(sqrt2}\OperatorTok{**}\DecValTok{2}\NormalTok{,}\FloatTok{2.0}\NormalTok{)}
\end{Highlighting}
\end{Shaded}

\begin{verbatim}
## True
\end{verbatim}

\begin{center}\rule{0.5\linewidth}{\linethickness}\end{center}

\begin{itemize}
\tightlist
\item
  floating point values are stored as a \textbf{mantissa} (digits) and
  an \textbf{exponent}
\end{itemize}

\begin{Shaded}
\begin{Highlighting}[]
\ImportTok{import}\NormalTok{ sys}
\NormalTok{sys.float_info()}
\end{Highlighting}
\end{Shaded}

\begin{itemize}
\tightlist
\item
  \texttt{max}=1.7976931348623157e+308 (the largest float that can be
  stored)
\item
  \texttt{max\_exp}=1024 (so 11 bits are needed to store the signed
  exponent)
\item
  \texttt{max\_10\_exp}=308
\item
  \texttt{min}=2.2250738585072014e-308 (closest to zero {[}almost{]})
\item
  \texttt{min\_10\_exp}=-307
\item
  \texttt{dig=15} (number of decimal digits)
\item
  \texttt{mant\_dig}=53 (bits in mantissa)
\item
  \texttt{epsilon=2.220446049250313e-16} (smallest number such that 1+x
  \textgreater{} x)
\end{itemize}

\hypertarget{overflow-and-underflow}{%
\subsection{overflow and underflow}\label{overflow-and-underflow}}

\begin{Shaded}
\begin{Highlighting}[]
\NormalTok{x }\OperatorTok{=} \FloatTok{1e308}
\NormalTok{small_x }\OperatorTok{=} \FloatTok{2e-323}
\BuiltInTok{print}\NormalTok{(}\StringTok{"x*1000="}\NormalTok{,x}\OperatorTok{*}\DecValTok{1000}\NormalTok{,}
      \StringTok{"}\CharTok{\textbackslash{}n}\StringTok{x*1000-x*1000="}\NormalTok{,x}\OperatorTok{*}\DecValTok{1000}\OperatorTok{-}\NormalTok{x}\OperatorTok{*}\DecValTok{1000}\NormalTok{,}
      \StringTok{"}\CharTok{\textbackslash{}n}\StringTok{small_x/1000"}\NormalTok{,small_x}\OperatorTok{/}\DecValTok{1000}\NormalTok{)}
\end{Highlighting}
\end{Shaded}

\begin{verbatim}
## x*1000= inf 
## x*1000-x*1000= nan 
## small_x/1000 0.0
\end{verbatim}

\texttt{inf} means ``infinity'' and \texttt{nan} means ``not a number''

\hypertarget{what-should-you-do-instead}{%
\subsection{What should you do
instead?}\label{what-should-you-do-instead}}

\begin{itemize}
\tightlist
\item
  devise a more stable algorithm (e.g.~one that adds items in increasing
  order)
\item
  work on the log scale (i.e.~add log values rather than multiplying
  values)
\item
  use extended/arbitrary precision floats: decimal module (built in), or
  mpmath
\item
  \textbf{always be careful comparing floating point}
\end{itemize}

\hypertarget{higher-precision}{%
\subsection{higher precision}\label{higher-precision}}

\begin{itemize}
\tightlist
\item
  temptation is just to increase precision

  \begin{itemize}
  \tightlist
  \item
    \texttt{float128} in numpy
  \item
    \texttt{mpmath} module for \textbf{arbitrary-precision} numbers (but
    infinite precision!)
  \end{itemize}
\end{itemize}

\begin{Shaded}
\begin{Highlighting}[]
\ImportTok{import}\NormalTok{ mpmath}
\BuiltInTok{print}\NormalTok{(}\OperatorTok{+}\DecValTok{1}\OperatorTok{*}\NormalTok{mpmath.pi)}
\end{Highlighting}
\end{Shaded}

\begin{verbatim}
## 3.14159265358979
\end{verbatim}

\begin{Shaded}
\begin{Highlighting}[]
\NormalTok{mpmath.mp.dps}\OperatorTok{=}\DecValTok{1000}
\BuiltInTok{print}\NormalTok{(}\OperatorTok{+}\DecValTok{1}\OperatorTok{*}\NormalTok{mpmath.pi)}
\end{Highlighting}
\end{Shaded}

\begin{verbatim}
## 3.141592653589793238462643383279502884197169399375105820974944592307816406286208998628034825342117067982148086513282306647093844609550582231725359408128481117450284102701938521105559644622948954930381964428810975665933446128475648233786783165271201909145648566923460348610454326648213393607260249141273724587006606315588174881520920962829254091715364367892590360011330530548820466521384146951941511609433057270365759591953092186117381932611793105118548074462379962749567351885752724891227938183011949129833673362440656643086021394946395224737190702179860943702770539217176293176752384674818467669405132000568127145263560827785771342757789609173637178721468440901224953430146549585371050792279689258923542019956112129021960864034418159813629774771309960518707211349999998372978049951059731732816096318595024459455346908302642522308253344685035261931188171010003137838752886587533208381420617177669147303598253490428755468731159562863882353787593751957781857780532171226806613001927876611195909216420198
\end{verbatim}

but you will often be disappointed



\end{document}
